%%%%%%%%%%%%%%%%%%%%%%%%%%%%%%%%%%%%%%%%%%%%%%%%%%%%%%%%%%%%%
%% Based on a TeXnicCenter-Template, which was             %%
%% created by Christoph Börensen                           %%
%% and slightly modified by Tino Weinkauf.                 %%
%%                                                         %%
%% Then, a third guy - me - put in some modifications.     %%
%%                                                         %%
%% MUS2323 - Plan de projet                                %%
%%%%%%%%%%%%%%%%%%%%%%%%%%%%%%%%%%%%%%%%%%%%%%%%%%%%%%%%%%%%%

\documentclass[letterpaper,12pt]{scrartcl}
% Optimised for letter. Add ",twosides" to use the two-sides layout.

% Margins
    \usepackage{vmargin}
    \setpapersize{USletter}
    \setmargins{2.0cm}%	 % Left edge
               {1.5cm}%  % Top edge
               {17.7cm}% % Text width
               {21.0cm}% % Text height
               {14pt}%	 % Header height
               {1cm}%    % Header distance
               {0pt}%	 % Footer height
               {2cm}%    % Footer distance
				
% Graphical bugfix (about footnotes)
    \usepackage[bottom]{footmisc}

% Fonts and locale
	\usepackage{t1enc}
	\usepackage[utf8]{inputenc}
	\usepackage{times}
	\usepackage[francais]{babel}
	\usepackage{amsmath}

	\AtBeginDocument {%
	    \renewcommand\tablename{\textsc{Tableau}}
	}

% Graphics
	\usepackage[pdftex]{graphicx}
	\usepackage{color}
	\usepackage{eso-pic}
	\usepackage{everyshi}
	\renewcommand{\floatpagefraction}{0.7}

% Enable hyperlinks
	\usepackage[pdfborder=000,pdftex=true]{hyperref}
	
% Table layout
	\usepackage{booktabs}

% Caption
	\usepackage{ccaption}
	\captionnamefont{\bf\footnotesize\sffamily}
	\captiontitlefont{\footnotesize\sffamily}
	\setlength{\abovecaptionskip}{0mm}

% Header and footer settings
	\usepackage{scrpage2} 
	\renewcommand{\headfont}{\footnotesize\sffamily}
	\renewcommand{\pnumfont}{\footnotesize\sffamily}

% Pagestyles
	\defpagestyle{cb}{
		(\textwidth,0pt) % Sets the border line above the header
		{\pagemark\hfill\headmark\hfill} % Doublesided, left page
		{\hfill\headmark\hfill\pagemark} % Doublesided, right page
		{\hfill\headmark\hfill\pagemark} % Onesided
		(\textwidth,1pt)} % Sets the border line below the header
		{(\textwidth,1pt) % Sets the border line above the footer
		{{\it Plan MUS2323}\hfill François Poitras} % Doublesided, left page
		{François Poitras\hfill{\it Plan MUS2323}} % Doublesided, right page
		{François Poitras\hfill{\it Plan MUS2323}} % One sided printing
		(\textwidth,0pt) % Sets the border line below the footer
	}

% Empty pages style
	\renewpagestyle{plain}
		{(\textwidth,0pt)
			{\hfill}{\hfill}{\hfill}
		(\textwidth,0pt)}
		{(\textwidth,0pt)
			{\hfill}{\hfill}{\hfill}
		(\textwidth,0pt)}

% Footnotes
	\renewcommand{\footnoterule}{\rule{5cm}{0.2mm} \vspace{0.3cm}}
	\deffootnote[1em]{1em}{1em}{\textsuperscript{\normalfont\thefootnotemark}}

\pagestyle{plain}

\begin{document}
	\begin{center}
		\vspace{2cm}

		{\Huge\bf\sf Plan de projet}

		\vspace{0.5cm}

		{\bf\sf (TP2)}

		\vspace{4cm}

		{\bf\sf Par}

		\vspace{0.5cm}{\large\bf\sf François Poitras}

		\vspace{2cm}

		{\bf\sf Présenté à}

		\vspace{0.5cm}{\large\bf\sf Olivier Bélanger}

		\vspace{2cm}

		{\bf\sf Dans le cadre du cours de}

		\vspace{0.5cm}{\large\bf\sf Création Musicale en Language Python 2 (MUS2323)}

		\vspace{\fill}
		Remis le \today

		\vspace{0.5cm}Université de Montréal
	\end{center}

	\newpage
	
	\section{Plan de travail}
	\subsection{Énoncé}
	
	\subsection{Analyse des besoins}
	\subsection{Acquisition de connaissances}
	Pyo, règles d'harmonie
	\subsection{Modèle}
	Une classe sera dédiée à contenir un "dictionnaire" de gammes et d'accords. Afin de faciliter l'abstraction, toutes les définitions seront basées sur la première octave MIDI. Par exemple, on pourra définir un accord majeur comme étant les notes $\{0,4,7\}$. La classe permettra de transposer --- par multiples de douze pour l'octave et en additionant ou soustrayant plus ou moins 11 pour la tonalité --- au besoin de l'utilisateur. Cette classe n'implémentera pas à proprement parler de processus audio, mais les classes utilisant de l'audio pourront utiliser ce dictionnaire comme référence. 
	
	Une autre classe aura le rôle de gérer l'interface graphique. Celle-ci sera composée d'une zone qui contiendra la progression d'accord (à la manière d'une grille d'accord). La version par défaut de la grille aura la forme d'un accord par mesure. Donc, si la progression dure 4 accords, la grille aura quatre cases. Il sera possible d'en ajouter ou d'en supprimer à l'aide d'un bouton. Par souci de simplicité, la version initiale à pour objectif de gérer seulement des mesures de 4/4. Le tempo sera quant à lui variable (avec une borne maximale et minimale).
	
	Une classe sera responsable d'implémenter différents instruments. Pour simplifier les choses, une classe mère aura les composantes principales et les classes enfants utiliseront l'héritage pour implémenter leur instrument. La classe mère aura entre autres des propriétés de volume, d'orientation sonore (gauche/droite) et de tempo. 
	\subsection{Méthodes}
	\subsection{Implémentation}


\end{document}