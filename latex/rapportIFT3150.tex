%%%%%%%%%%%%%%%%%%%%%%%%%%%%%%%%%%%%%%%%%%%%%%%%%%%%%%%%%%%%%
%% Based on a TeXnicCenter-Template, which was             %%
%% created by Christoph Börensen                           %%
%% and slightly modified by Tino Weinkauf.                 %%
%%                                                         %%
%% Then, a third guy - me - put in some modifications.     %%
%%                                                         %%
%%                IFT3150 - Rapport                        %%
%%%%%%%%%%%%%%%%%%%%%%%%%%%%%%%%%%%%%%%%%%%%%%%%%%%%%%%%%%%%%

\documentclass[letterpaper,12pt]{scrartcl}
% Optimised for letter. Add ",twosides" to use the two-sides layout.

% Margins
    \usepackage{vmargin}
    \setpapersize{USletter}
    \setmargins{2.0cm}%	 % Left edge
               {1.5cm}%  % Top edge
               {17.7cm}% % Text width
               {21.0cm}% % Text height
               {14pt}%	 % Header height
               {1cm}%    % Header distance
               {0pt}%	 % Footer height
               {2cm}%    % Footer distance
				
% Graphical bugfix (about footnotes)
    \usepackage[bottom]{footmisc}

% Fonts and locale
	\usepackage{t1enc}
	\usepackage[utf8]{inputenc}
	\usepackage{times}
	\usepackage[francais]{babel}

	\AtBeginDocument {%
	    \renewcommand\tablename{\textsc{Tableau}}
	}

% Graphics
	\usepackage[pdftex]{graphicx}
	\usepackage{color}
	\usepackage{eso-pic}
	\usepackage{everyshi}
	\renewcommand{\floatpagefraction}{0.7}

% Enable hyperlinks
	\usepackage[pdfborder=000,pdftex=true]{hyperref}
	
% Table layout
	\usepackage{booktabs}

% Caption
	\usepackage{ccaption}
	\captionnamefont{\bf\footnotesize\sffamily}
	\captiontitlefont{\footnotesize\sffamily}
	\setlength{\abovecaptionskip}{0mm}

% Header and footer settings
	\usepackage{scrpage2} 
	\renewcommand{\headfont}{\footnotesize\sffamily}
	\renewcommand{\pnumfont}{\footnotesize\sffamily}

% Pagestyles
	\defpagestyle{cb}{
		(\textwidth,0pt) % Sets the border line above the header
		{\pagemark\hfill\headmark\hfill} % Doublesided, left page
		{\hfill\headmark\hfill\pagemark} % Doublesided, right page
		{\hfill\headmark\hfill\pagemark} % Onesided
		(\textwidth,1pt)} % Sets the border line below the header
		{(\textwidth,1pt) % Sets the border line above the footer
		{{\it Plan MUS2323}\hfill François Poitras} % Doublesided, left page
		{François Poitras\hfill{\it Plan MUS2323}} % Doublesided, right page
		{François Poitras\hfill{\it Plan MUS2323}} % One sided printing
		(\textwidth,0pt) % Sets the border line below the footer
	}

% Footnotes
	\renewcommand{\footnoterule}{\rule{5cm}{0.2mm} \vspace{0.3cm}}
	\deffootnote[1em]{1em}{1em}{\textsuperscript{\normalfont\thefootnotemark}}

% Document specific
\usepackage{hyperref}
\usepackage{amssymb}
\pagestyle{empty}

\begin{document}
	\begin{center}
		\vspace{2cm}

		{\Huge\bf\sf Rapport de projet}
		\vspace{4cm}

		{\bf\sf Par}

		\vspace{0.5cm}{\large\bf\sf François Poitras}

		\vspace{2cm}

		{\bf\sf Présenté à}

		\vspace{0.5cm}{\large\bf\sf Olivier Bélanger et Sylvie Hamel}

		\vspace{2cm}

		{\bf\sf Dans le cadre du cours de}

		\vspace{0.5cm}{\large\bf\sf Projet d'informatique (IFT3150)}

		\vspace{\fill}
		
		\today

		\vspace{0.5cm}
		Université de Montréal
	\end{center}

	\newpage
	\pagestyle{plain}
	\section{Plan de travail initial}
	
	\subsection{Énoncé}
	Il s'agit de complémenter une application qui permet actuellement à l'utilisateur de construire des progressions d'accords arbitraires. Le but du projet est de permettre au programme de construire une mélodie agréable qui accompagne les accords. Pour ce faire, il faut tout d'abord déterminer quelle est la tonalité de la progression. Ensuite, il faut établir quelle(s) gammes peuvent être utilisées, en se basant sur les notes des accords présents. Vient ensuite la partie la plus difficile, celle où il faut développer des idées musicales de manière pseudo-réaliste et non simplement jouer des notes de la gamme au hasard. Bien sur, il y aura une partie de hasard dans le choix des notes ou des phrases, mais l'idée est d'implémenter des structures qui permettent de générer une mélodie "écoutable". Pour ce faire, différentes avenues seront explorées, en se basant sur les lectures préliminaires. 

Pour ce qui est de la basse, la technique devrait d'être plus simple. On peut simplement implémenter quelques recettes et les varier au fil des mesures. Par exemple, on peut jouer la fondamentale sur le temps 1, jouer une note de l'accord sur le temps 2 et jouer une note différente sur le temps 3. Pour le quatrième temps, on approche l'accord suivant, par exemple avec du chromatisme. 

Si le temps le permet, des synthétiseurs additionels seront développés pour varier l'instrumentalisation du programme, autant pour les accords que pour la mélodie ou la basse. Pour aider au développement d'éventuels synthétiseurs, la lecture de la série "Synth Secrets" de la revue Sound on Sound pourra aider. Ce développement se fera en parallèle des étapes mentionnées dans l'échéancier, il ne consiste pas en une étape en tant que tel. 

Note : L'application a déjà été commencée dans le cadre du cours MUS3327 à l'Hiver 2017. La partie sur les progressions d'accord a été développée dans ce cours. Au moment de débuter le projet, l'application génère également une ligne de basse plutôt primitive. Aucun calcul n'est fait sur celle-ci, elle ne fait que suivre les accords. Tout le code est disponible à l'adresse suivante : 

\href{https://github.com/poifra/Creamuspython2}{https://github.com/poifra/Creamuspython2}. Il s'agit du dépôt utilisé pour le cours MUS3327 avec les fonctionalités supplémentaires développées dans le cadre du projet.

	\subsection{Résumé du développement}
	\subsubsection{Échéancier}
	\begin{itemize}
	\item 7 mai	\\
	J'ai commencé la recherche d'articles pour avoir une idée de l'approche à utiliser. Plusieurs articles utilisent des processusus stochastiques comme des chaînes de Markov pour construire des séquences de notes intéressantes. Une autre approche consiste à considérer les règles d'harmonies musicale comme une grammaire d'une langue naturelle. Ainsi, on peut appliquer des techniques de traitement de langue à des «phrases» musicales, tout en gardant nécéssairement une composante aléatoire. Ce sont les deux méthodes qui semblent le plus prometteuses. Une troisième approche serait d'utiliser des techniques d'apprentissage machine pour générer des mélodies, mais considérant le temps disponible pour réaliser le projet, cette approche ne sera pas explorée. En effet, le temps d'apprentissage semble plutôt long et l'implémentation demanderait aussi trop de temps. 

Les deux méthodes mentionnées semblent toutefois avoir l'inconvénient de se baser sur du matériel existant pour établir les transitions enre les notes. Des recherches plus approfondies seront nécéssaires pour déterminer dans quel mesure il est possible de créer une improvisation à partir uniquement d'une séquence d'accords 

Je n'ai pas terminé de lire tous les articles, mais ceux-ci sont les plus intéressants que j'ai trouvé:
\href{http://wesscholar.wesleyan.edu/cgi/viewcontent.cgi?article=1344&context=etd_hon_theses}{A Clustering Algorithm for Recombinant Jazz Improvisations}\\
\href{http://peterlangston.com/Papers/amc.pdf}{Six Techniques for Algorithmic Music Composition} \\
\href{https://books.google.ca/books?id=jaowAtnXsDQC&pg=PA118&lpg=PA118&dq=algorithmic+improvisation+grammar&source=bl&ots=GNMbfN7fZr&sig=VptvrQhvYNoqS1yv5XBOlrpY91M&hl=fr&sa=X&ved=0ahUKEwjL_b-qn9TTAhVr_4MKHe_wBBgQ6AEIOTAC#v=onepage&q=algorithmic\%20improvisation\%20grammar&f=false}{Paradigms of Automated Music Generation}\\

Et un dernier, un peu plus ésotérique mais pouvant être intéressant : \href{http://quod.lib.umich.edu/cgi/p/pod/dod-idx/algorithmic-musical-improvisation-from-2d-board-games.pdf?c=icmc;idno=bbp2372.2010.061}{Algorithmic Musical Improvisation From 2d Board Games}\\
	\item 20 mai \\
	En raison d'un problème de santé, la rencontre prévue avec Olivier n'a pas pu avoir lieu et donc le travail n'a pas pu débuter aussi rapidement que prévu. Nous essayons présentement de trouver une autre date. J'ai également décidé de ne pas développer d'algorithme permettant la reconnaissance d'une tonalité, à partir d'une séquence d'accords. Bien qu'il soit théoriquement possible de le faire, il existe trop d'ambiguités qu'un programme de peut pas résoudre. En effet, un musicien pourrait décider d'utiliser une séquence d'accords qui n'harmonise aucune gamme, par exemple : G - B - C - Cm. Cette progression est utilisée dans la chanson \textit{Creep} de Radiohead. On remarquera qu'il y a un Do majeur et un Do mineur, ce qui rend une analyse simple impossible. Je vais plutôt ajouter une boîte de texte dans l'application qui demande à l'utilisateur quelle tonalité utiliser. De cette façon, même si il y a des "accidents", le programme aura une idée de quelle tonalité utiliser. La programmation d'une ligne de basse mélodique risque de prendre un peu de retard, mais elle devrait commencer sous peu. L'horaire du projet a aussi été revu, pour prendre en compte le fait que la présentation aura lieu plus tard que prévu initialement.
	\item 1er juin\\
	Après une rencontre avec Olivier, nous avons pu établir les grandes orientations du projet. Premièrement, le moteur sera développé de façon séparée de l'interface, afin qu'il soit éventuellement inclu dans pyo-tools. Olivier m'a également suggéré la lecture du livre que j'avais trouvé dans mes recherches initiales (voir premier rapport). 

Suite aux discussions, j'ai décidé d'utiliser une approche intrinsèquement aléatoire dans la génération de mélodie. L'autre méthode serait de rechercher des banques MIDI afin d'utiliser des chaînes de Markov. Olivier m'a suggéré d'étudier le fonctionnement des classes de pyo qui utilisent de l'aléatoire afin de m'en insipirer. 
J'ai également commencé à réfléchir à la structure que je veux utiliser pour la "walking bass". Celle-ci va utiliser des règles plutôt fixes qui vont faire en sorte que la mélodie de basse va bien fonctionner. Le programme ne devrait pas souvent gébérer deux fois la même ligne, mais celles-ci risquent de se ressembler.
	\item 14 juin\\
	La walking bass est fonctionnelle. Celle-ci est une version à peine plus complexe de la version d'origine. La basse joue uniquement des noires, donc quatre notes par mesure. Le premier temps est la tonique, c'est-à-dire la note fondamentale de l'accord courant. Le deuxième temps est une note aléatoire de la gamme (mineure ou majeure, selon le cas) correspondante. Le troisième temps est une des notes composant l'accord (par exemple la tierce, la quinte ou la septième) choisie aléatoirement. Finalement, le dernier temps est une note d'approche, c'est à dire une note qui est un demi-ton sous la tonique de l'accord suivant. Des améliorations sont possibles. Par exemple, l'algorithme pourrait s'assurer de ne pas jouer deux fois la même note de façon consécutive, pour que la mélodie de basse soit plus intéressante. Aussi, lors de tests, j'ai remarqué que la gestion du tempo n'était pas bien implémentée et devra être corrigée dans la version finale. 
	\item 28 juin\\
	J'ai implémenté différents générateurs aléatoires. Ils sont basés sur les générateurs présents dans pyo. La prochaine étape consiste à tester les différentes distributions (uniforme, gaussienne, Poisson, etc.) implémentées pour voir si certaines produisent des résultats plus intéressants. Je ne suis pas certain de la méthode que je veux adopter pour relier les générateurs à des notes MIDI, mais la librairie possède une fonction snap qui a un comportement similaire. Il est possible que je m'en inspire. Actuellement, tous les générateurs donnent un flottant compris dans l'interval [0, 1[. En raison des retards dans l'horaire prévu, il est possible que la programmation de mélodies aléatoires ne soit pas terminée selon l'échéancier initial.
	\item 10 juillet\\
	J'éprouve quelques difficultés à contacter Olivier pour lui poser des questions sur l'orientation du projet. Entre autres, l'implémentation que j'ai utilisée pour les générateurs semble extrêmement biaisée. Pour l'instant, je teste les différents générateurs avec une fonction snap qui fait en sorte de relier une distribution de valeurs à une gamme quelquoncque. Toutefois, les résultats ne sont pas très intéressants au niveau musical. Il pourrait être intéressant de développer des générateurs plus complexes, qui font appel à des motifs musicaux générés précédemment. Il y a aussi l'aspect rythmique de la mélodie qui n'a pas encore été travaillé. Il pourrait être possible d'implémenter les motifs rythmiques les plus communs et de tenter de les combiner à la génération aléatoire.
	\item 24 juillet\\
	Je suis présentement en train de relier les générateurs aléatoires au moteur audio développé plus tôt au cours du projet. L'idée est la même que pour la walking bass, c'est à dire que l'accord courant est utilisé comme base. Un ensemble de notes est généré, basé sur une distribution aléatoire. Il va bientôt être possible (d'ici environ une semaine) de juger de la qualité de l'improvisation ainsi générée. Cependant, je pense qu'il va falloir appliquer d'autres techniques pour avoir un résultat décent. Par exemple, il pourrait être possible de ré-utiliser certains passages générés précédemment et de les varier un peu. On peut aussi biaiser un générateur aléatoire afin qu'il produise plus de notes qui harmonisent l'accord courant, plutôt que simplement des notes d'une gamme.
	
	\item 7 août
	
	\item 18 août\\
	Envoi du rapport.
	
	\end{itemize}

	\section{Résultats}
	La ligne de basse est correcte. Elle est fixée à un tempo qui est le double de celui des accords. Comme les accords sont des blanches, le tempo est donné par la basse. Ceci a été fait afin de simplifier le processus de synthèse. En donnant la gestion du tempo à la basse, il est ainsi possible d'éviter la synthèse d'une batterie, qui aurait nécéssité du temps sans rien apporter aux résultats mélodiques. Toutefois, il est évidemment possible d'améliorer la basse, mais elle joue son rôle efficacement dans l'implémentation actuelle. Pour plus de détails concernant les améliorations, voir la section sur les perspectives. 

	L'élément qui a été plus complexe à modéliser est la mélodie proprement dite. En effet, créer une mélodie qui respecte une tonalité est plutôt simple en soi. La base de la génération est basée sur différentes distributions aléatoires, qui, pour faciliter leur traitement subséquent, sont $\mathbb{R} \in [0,1]$. Tous les générateurs sont définis dans la classe \textit{MyRandoms}. La distribution la plus simple est uniforme et sert de base pour les autres. Évidemment, cette distribution est pseudo-aléatoire, mais la distinction est négligeable dans ce projet, puisqu'il n'y a pas de contraintes de sécurité.
	
	\subsubsection{Générateurs particuliers}
	Deux générateurs ne se qualifient pas tout à fait d'aléatoires. Il s'agit des modes \textit{walker} et \textit{loopseg}.
	\subsection{Qualité de l'improvisation mélodique}
	L'appréciation de l'improvisation est évidemment subjective, en autant que la mélodie soit dans la même tonalité que la progression d'accord. 
	\subsection{Problèmes rencontrés}
	
	\section{Perspectives}
		\subsection{Mélodie}

		\subsection{Basse}
	
		\subsection{Style musical}

		\subsection{Éléments graphiques}
		L'interface graphique en tant que tel est assez intuitive à utiliser, mais la construction elle même de progression d'accords pourrait être révisée. Dans la version actuelle, les accords sont représentés par une pile. Pour remplacer le premier accord de la progression, il faut donc dépiler tous les accords avant de pouvoir modifier le premier.
		
		 La solution la plus simple serait de demander à l'usager à quelle position il veut placer l'accord qu'il a choisit. Par contre, il pourrait être difficile d'implémenter cette fonction sans alourdir considérablement l'interface. Une solution plus complexe est d'implémenter un système de \textit{drag and drop}, tel que mentionné dans le plan initial. De cette façon, l'utilisateur pourrait déplacer à sa guise les accords qu'il a placé et cela garderait l'esprit intuitif de l'interface.
\end{document}